\chapter{Dirac Equation}

\section{Introduction}

The Dirac equation for one particle is \cite{strange,zabloudil}:
$$H\psi=W\psi\,,\no{diraceq}$$
$$H=c\balpha\cdot{\bf p}+\beta mc^2+V(r)\hbox{\dsrom1}\,,$$
where $\psi$ is a four component vector:
$$\psi=\left(\matrix{\psi_1\cr\psi_2\cr\psi_3\cr\psi_4\cr}\right)
=\col{\psi_A}{\psi_B}\,,\qquad
\psi_A=\col{\psi_1}{\psi_2}\!,\,\psi_B=\col{\psi_3}{\psi_4}$$
and $\balpha$, $\beta$ are $4\times4$ matrices:
$$\balpha=\mat{0}{\bsigma}{\bsigma}{0}\,,$$
$$\beta=\mat{\hbox{\dsrom1}}{0}{0}{-\hbox{\dsrom1}}\,,$$
where the Pauli matrices $\bsigma=(\sigma_x,\sigma_y,\sigma_z)$ and
$\hbox{\dsrom1}$ form a basis of all $2\times2$ Hermitian matrices.
Substituting all of this into \rno{diraceq} yields: 
$$c\,\bsigma\cdot{\bf p}\col{\psi_B}{\psi_A}=
\matd{W-V-mc^2}{W-V+mc^2}\col{\psi_A}{\psi_B}\,.\no{dirac2}$$
To derive a continuity equation, we multiply \rno{diraceq} by $\psi^*$
and subtract the conjugate transpose of \rno{diraceq} multiplied by $\psi$:
$$\p{}{t}(\psi^*\psi)=-\nabla\cdot(c\psi^*\balpha\psi)\,,$$
so we identify the probability and current densities as
$$\rho=\psi^*\psi=\psi_1^*\psi_1+\psi_2^*\psi_2+\psi_3^*\psi_3+\psi_4^*\psi_4\,,
\qquad {\bf j}=c\psi^*\balpha\psi\,.$$
The normalization of a four-component wave function is then
$$
\int \rho \,\d^3x=
\int \psi^*\psi \,\d^3x=
\int \psi_1^*\psi_1+\psi_2^*\psi_2+\psi_3^*\psi_3+\psi_4^*\psi_4 \,\d^3x=
1\,.\no{norm}$$
The probability density $\rho(x,y,z)$ is the physical quantity we are
interested in, and all the four-component wavefunctions and other formalism is
just a way of calculating it. This $\rho$ is also the thing we should compare
with the Schr\"odinger equation. 

\section{Derivation of the radial equation}

We rewrite $\bsigma\cdot{\bf p}$ \cite{branson}:
$$\bsigma\cdot{\bf p}={1\over r}{\bsigma\cdot{\bf x}\over r}
\left(-i\hbar r \p{}{r}+i\bsigma\cdot{\bf L}\right),\no{sigmap}$$
and search for a basis in the form
$$\psi_A=g\chi^{j_3}_{\kappa}\,,\no{psia}$$
$$\psi_B=if\chi^{j_3}_{-\kappa}\,.\no{psib}$$
We chose this form, because we want $\psi_A$ and $\psi_B$ to have the same $j$
and $j_3$, so the only thing they can differ in is $l$ (the two allowed
possibilities are $l=j\pm\half$). According to \rno{chi1}, \rno{chi2},
\rno{chi3} and \rno{chi2}, this corresponds to $\pm\kappa$ (see also
\cite{kellog,strange,rose,branson} and the appendix for more information about
spin-angular functions).

Substituting \rno{psia}, \rno{psib} and \rno{sigmap} into \rno{dirac2} we get:
$$c{1\over r}{\bsigma\cdot{\bf x}\over r} \left(-i\hbar r
\p{}{r}+i\bsigma\cdot{\bf
L}\right)\col{if\chi^{j_3}_{-\kappa}}{g\chi^{j_3}_{\kappa}}=
\hskip3cm
$$ 
$$
\hskip3cm
=
\matd{W-V-mc^2}{W-V+mc^2}\col{g\chi^{j_3}_{\kappa}}{if\chi^{j_3}_{-\kappa}}\,,$$
Both $\psi_A$ and $\psi_B$ are eigenvectors of the operator 
$K=\beta(\bsigma\cdot{\bf L}+\hbar)$:
$$K\psi_{A,B}=-\hbar\kappa\psi_{A,B}\,,$$
so the action of the $\bsigma\cdot{\bf L}$ operator on the
$\psi_A$ and $\psi_B$ can easily be determined:
$$\bsigma\cdot{\bf L}\col{\psi_A}{\psi_B}=
(\beta K-\hbar)\col{\psi_A}{\psi_B}=
\hskip3cm
$$ 
$$
\hskip3cm
=
\matd{K-\hbar}{-K-\hbar}\col{\psi_A}{\psi_B}=
\col{\hbar(-\kappa-1)\psi_A}{\hbar(\kappa-1)\psi_B}$$
and we can write
$$c{1\over r}{\bsigma\cdot{\bf x}\over r} 
\col{(-i\hbar r\p{}{r}+i(\kappa-1)\hbar)if\chi^{j_3}_{-\kappa}}
{(-i\hbar r\p{}{r}+i(-\kappa-1)\hbar)g\chi^{j_3}_{\kappa}}=
\hskip3cm
$$ 
$$
\hskip3cm
=
\matd{W-V-mc^2}{W-V+mc^2}\col{g\chi^{j_3}_{\kappa}}{if\chi^{j_3}_{-\kappa}}\,.$$
The operator ${\bsigmasmall\cdot{\bf x}\over r}$ also only acts on the angular
momentum parts of the state \cite{strange} (page 59). 
From \rno{parity:appendix}:
$${\bsigma\cdot{\bf x}\over r}\chi^{j_3}_{\kappa}=-\chi^{j_3}_{-\kappa}\,,$$
$${\bsigma\cdot{\bf x}\over r}\chi^{j_3}_{-\kappa}=-\chi^{j_3}_{\kappa}\,,$$
so
$$c{1\over r} 
\col{(i\hbar r\p{}{r}-i(\kappa-1)\hbar)if\chi^{j_3}_{\kappa}}
{(i\hbar r\p{}{r}+i(\kappa+1)\hbar)g\chi^{j_3}_{-\kappa}}=
\hskip3cm
$$ 
$$
\hskip3cm
=
\matd{W-V-mc^2}{W-V+mc^2}\col{g\chi^{j_3}_{\kappa}}{if\chi^{j_3}_{-\kappa}}\,,$$
rewriting
$$\hbar c{1\over r} 
\matd{(-r\p{}{r}+(\kappa-1))f}
{(r\p{}{r}+(\kappa+1))g}\col{\chi^{j_3}_{\kappa}}{i\chi^{j_3}_{-\kappa}}=
\hskip3cm$$
$$\hskip3cm=\matd{(W-V-mc^2)g}{(W-V+mc^2)f}
\col{\chi^{j_3}_{\kappa}}{i\chi^{j_3}_{-\kappa}}\,,$$
and canceling the same terms on both sides finally yields
$$\hbar c
\col{-\p{f}{r}+{\kappa-1\over r}f}
{\p{g}{r}+{\kappa+1\over r}g}=
\col{(W-V-mc^2)g}{(W-V+mc^2)f}\,.\no{radialdirac}$$
This is the radial Dirac equation. As we shall see in the next section, the
equation for $g$ is (with the exception of a few relativistic corrections)
identical to the radial Schr\"odinger equation. And $f$ vanishes
in the limit $c\to\infty$. For this reason $f$ is called the small 
(fein, minor) component and $g$ the large (gro\ss, major) component. 

The probability density is
$$\rho=\psi^*\psi=\psi^*_A\psi_A+\psi^*_B\psi_B=
f^2\chi^{j_3*}_{-\kappa}\chi^{j_3}_{-\kappa}+
g^2\chi^{j_3*}_{\kappa}\chi^{j_3}_{\kappa}\,,
$$
so from the normalization condition \rno{norm} we get
$$
\int \rho \,\d^3x=
\int f^2\chi^{j_3*}_{-\kappa}\chi^{j_3}_{-\kappa}+
g^2\chi^{j_3*}_{\kappa}\chi^{j_3}_{\kappa} \,\d^3x=
\int (f^2\chi^{j_3*}_{-\kappa}\chi^{j_3}_{-\kappa}+
g^2\chi^{j_3*}_{\kappa}\chi^{j_3}_{\kappa})\,r^2\,\d r\d\Omega=
$$
$$
=\int_0^\infty\!\!\!\!\!f^2r^2\,\d r\int\chi^{j_3*}_{-\kappa}\chi^{j_3}_{-\kappa}
\,\d\Omega+
\int_0^\infty\!\!\!\!\!g^2r^2\,\d r\int\chi^{j_3*}_{\kappa}\chi^{j_3}_{\kappa}
\,\d\Omega=
\int_0^\infty r^2(f^2+g^2)\,\d r=1\,,
$$
where we used the normalization of spin-angular functions \rno{spinnorm}.
Also it can be seen, that the radial probability density
is 
$$\rho(r)=r^2(f^2+g^2)$$
(i.e., the probability to find the electron
between $r_1$ and $r_2$ is $\int_{r_1}^{r_2}r^2(f^2+g^2)\,\d r$). In the 
nonrelativistic case, the density is given by
$$\rho(r)=r^2R^2\,,$$
so the correspondence between the Schr\"odinger and Dirac equation is 
$R^2=f^2+g^2$. 


\section{Numerical integration for a given $E$}

We rewrite \rno{radialdirac} to a form found in \cite{strange} (eq. 8.10 and
8.12):
$$\eqalignno{
g_\kappa'&=-{\kappa+1\over r}g_\kappa+{1\over c\hbar}(W-V+mc^2)f_k\,,\nno{strange1}\cr
f_\kappa'&=+{\kappa-1\over r}f_\kappa-{1\over c\hbar}(W-V-mc^2)g_k\,.\nno{strange2}\cr
}$$
Let $\hbar=1$ and define $E=W-mc^2$, so that $E$ doesn't contain the electron
rest mass energy. Next we define a relativistic mass
$$M=m+{1\over2c^2}(E-V)\no{relmass}$$
and we get 
$$\eqalignno{
g_\kappa'&=-{\kappa+1\over r}g_\kappa+2Mcf_k\,,\nno{gkappa}\cr
f_\kappa'&=+{\kappa-1\over r}f_\kappa+{1\over c}(V-E)g_k\,,\nno{fkappa}\cr
}$$
which are the same equations as in \cite{koelling&harmon} (eq. 2a and 2b).

From \rno{gkappa} we express $f_\kappa$ and substitute it into \rno{fkappa}.
At the same time we introduce a new variable
$$\phi_\kappa={1\over2M}g_\kappa'\,,\no{phikappa}$$
(beware, \cite{koelling&harmon} introduce $\phi_\kappa={1\over2Mc}g_\kappa'$)
to simplify our equations:
$$f_\kappa={g_\kappa'\over2Mc}+{\kappa+1\over r}{g_\kappa\over2Mc}
={\phi_\kappa\over c}+{\kappa+1\over c}{g_\kappa\over2Mr}\,,\no{fk}$$
differentiate
$$f_\kappa'=
{\phi_\kappa'\over c}+{\kappa+1\over c}\left(g_\kappa\over2Mr\right)'
={\phi_\kappa'\over c}+{\kappa+1\over cr}\phi_\kappa+
{\kappa+1\over2M^2cr}g_\kappa{V'\over2c^2}-{\kappa+1\over2Mcr^2}g_\kappa$$
and substitute $f_\kappa$ and $f_\kappa'$ back to \rno{fkappa}:
$$
{\phi_\kappa'\over c}+{\kappa+1\over cr}\phi_\kappa+
{\kappa+1\over2M^2cr}g_\kappa{V'\over2c^2}-{\kappa+1\over2Mcr^2}g_\kappa=$$
$$={\kappa-1\over r}
\left(
{\phi_\kappa\over c}+{\kappa+1\over c}{g_\kappa\over2Mr}
\right)
+{1\over c}(V-E)g_k\,,$$
by a simplification we finally get
$$
\phi_\kappa'(r)=-{2\over r}\phi_\kappa(r)+\left[\big(V(r)-E\big)+
{\kappa(\kappa+1)\over 2M(r)r^2}-
{\kappa+1\over4M^2(r)c^2r}V'(r)\right]g_\kappa(r)\,.\no{phideriv}
$$ 
Equations \rno{phideriv}, \rno{phikappa} and \rno{relmass} are the equations
we are solving. It is instructive to write equation for $g_\kappa$
directly. For this, we need to calculate 
$$\phi_\kappa'={g''\over2M}-{g'M'\over2M^2}$$
and substituting for $\phi_\kappa$ and $\phi_\kappa'$ into \rno{phideriv} gives
$${g_\kappa''\over2M}-{g_\kappa'M'\over2M^2}=-{2\over
r}{g_\kappa'\over2M}+\left[\big(V-E\big)+ {\kappa(\kappa+1)\over 2Mr^2}-
{\kappa+1\over4M^2c^2r}V'\right]g_\kappa\,,$$
multiplying by $2M$ and rewriting
$${g_\kappa''}=-\left({2\over r}-{M'\over M}\right)
g_\kappa'+\left[\big(V-E\big)+ {\kappa(\kappa+1)\over 2Mr^2}-
{\kappa+1\over4M^2c^2r}V'\right]2Mg_\kappa\,.$$
From \rno{relmass} it follows
$${M'\over M}=-{V'\over2Mc^2}\,,$$
so we finally get
$${g_\kappa''}=-\left({2\over r}+{V'\over2Mc^2}\right)
g_\kappa'+\left[\big(V-E\big)+ {\kappa(\kappa+1)\over 2Mr^2}-
{\kappa+1\over4M^2c^2r}V'\right]2Mg_\kappa\,.\no{eqgk}$$
Comparing \rno{eqgk} with \rno{radial} we clearly see the two relativistic
corrections, both depending on $V'$ and both vanishing for $c\to\infty$ as
expected. Also it should be noted that there is another difference, that
\rno{eqgk} contains the relativistic mass \rno{relmass}, but \rno{radial}
contains the rest mass $m$.

All the terms in the \rno{eqgk} are the same for both possibilities
$j=l\pm\half$ (i.e. $\kappa(\kappa+1)=l(l+1)$ for
both $\kappa=-l-1$ and $\kappa=l$), 
except for the spin-orbit coupling
term ${\kappa+1\over4M^2c^2r}V'(r)$, see for example \cite{strange}
(eq. 2.64).
Sometimes it makes sense to
consider a semirelativistic case, where we neglect the
spin-orbit coupling term, in which case we are left with the
Schr\"odinger equation with the relativistic mass $M$ and only one
correction ${V'\over2Mc^2}$. 

In practice, the potential $V$ is given on a discrete grid, so we need to
compute it's derivative $V'$ numerically. This is the reason we solve 
\rno{phideriv} instead of \rno{eqgk}, because we need to evaluate $V'$ only
once for the spin-orbit term (for the semirelativistic case we don't even need
the $V'$ at all). But besides this minor technical thing, there is no other
reason we chose \rno{phideriv} and not \rno{eqgk}, which looks more familiar.

Once we have calculated $g_\kappa$ and $\phi_\kappa$, we calculate $f_\kappa$ 
from \rno{fk}. So the result of the radial
Dirac equation are two functions $f_\kappa$ and $g_\kappa$. The
physically relevant quantity is the radial probability density
$$\rho(r)=r^2(f^2_\kappa+g^2_\kappa)\,.$$

We calculate the functions $f_\kappa$ and $g_\kappa$ in a similar
way as we calculated $R$ for the Schr\"odinger equation, thus we 
need the asymptotic behavior at the origin. The potential
can always be treated as $V=1/r+\cdots$ and in this case 
it can be shown \cite{zabloudil}, that the asymptotic is 
$$g_\kappa=r^{\beta-1}\,,$$
$$\phi_\kappa={(\beta-1)r^{\beta-2}\over2M}\,,$$
where 
$$\beta=\sqrt{\kappa^2-\left(Z\over c\right)^2}\,,\no{diracasymptotic}$$
or, if we write it explicitly, for $j=l+\half$
$$\beta^+=\sqrt{(-l-1)^2-\left(Z\over c\right)^2}$$
and $j=l-\half$
$$\beta^-=\sqrt{l^2-\left(Z\over c\right)^2}\,.$$
In the semirelativistic case (which is an approximation --- we neglect
the spin-orbit coupling term) we choose
$$\beta=\sqrt{\half(|\beta^+|^2+|\beta^-|^2)}=
\sqrt{l^2+l+\half-\left(Z\over c\right)^2}\,.$$
It should be noted that in the literature we can find other types of 
aymptotic behaviour for the semirelativistic case, its just a question of the
used approximation. One can hardly say that some of them are correct and
another is not since the semirelativistic (sometimes denoted as
scalar-relativstic) approximation itself is not correct, it's just an
approximation.

It follows from \rno{diracasymptotic} that for $j=l+\half$ the radial Dirac
equation completely becomes the radial Schr\"odinger equation in the limit
$c\to\infty$ (and gives exactly the same solutions). For $j=l-\half$ however, 
we get a wrong asymptotic: we get a radial Schr\"odinger equation for $l$, but
the asymptotic for $l-1$.


\section{Other forms of the equations}

Unfortunately, there are a lot of different forms the radial Dirac equation can
be found in the literature. Many authors use different symbols, different
units, I even found a mistake in \cite{donald:apw}. For the reader's comfort,
this section is devoted to deriving and presenting the most common forms of the
equations. 

We realize the fact
$$\eqalign{
f_\kappa'-{\kappa-1\over r}f_\kappa&=
{1\over r}\left({\d\over\d r}-{\kappa\over r}\right)(rf_\kappa)\,,\cr
g_\kappa'+{\kappa+1\over r}g_\kappa&=
{1\over r}\left({\d\over\d r}+{\kappa\over r}\right)(rg_\kappa)\,,\cr
}$$
and rewrite equations \rno{strange1} and \rno{strange2}:
$$\eqalign{
{1\over r}\left({\d\over\d r}+{\kappa\over r}\right)(rg_\kappa)
&-{1\over c}(W-V+mc^2)f_k=0\,,\cr
{1\over r}\left({\d\over\d r}-{\kappa\over r}\right)(rf_\kappa)
&+{1\over c}(W-V-mc^2)g_k=0\,.\cr
}$$
These equations could be found in \cite{zabloudil} (eq. 8.10 and 8.9).
Let's make the substitution \cite{donald:apw}
$$\eqalign{
P_\kappa&=rg_\kappa\,,\cr
Q_\kappa&=rf_\kappa\cr
}$$
and write
$$\eqalign{
\left({\d\over\d r}+{\kappa\over r}\right)P_\kappa
&-{1\over c}(W-V+mc^2)Q_k=0\,,\cr
\left({\d\over\d r}-{\kappa\over r}\right)Q_\kappa
&+{1\over c}(W-V-mc^2)P_k=0\,,\cr
}$$
which can be found in \cite{engel} (eq. 3 and 4: they write 
$a_{nlj}(r)\equiv P_\kappa$ and $b_{nlj}(r)\equiv Q_\kappa$) and also
in \cite{strange} (eq. 8.13: he uses
$u_\kappa\equiv P_\kappa$ and $v_\kappa\equiv Q_\kappa$).

Now we use \rno{relmass} and these relations become
$$\eqalign{
W-V+mc^2&=E-V+2mc^2=2Mc^2\,,\cr
W-V-mc^2&=E-V\,,\cr
}$$
to write
$$\eqalign{
{\d P_\kappa\over\d r}&=-{\kappa\over r}P_\kappa+2McQ_k\,,\cr
{\d Q_\kappa\over\d r}&={\kappa\over r}Q_\kappa-{1\over c}(E-V)P_k\,,\cr
}$$
which can be found in \cite{donald:apw} (eq. 2a and 2b) (there is a mistake
there, they forgot to divide by $r$). $2M$ can be written explicitly as
$$2M=\left[{E-V\over c^2}+2m\right]=\left[{E-V\over c^2}+2\right]\rm\,a.u.\,,$$
so we get
$$\eqalign{
{\d P_\kappa\over\d r}&=-{\kappa\over r}P_\kappa+\left[{E-V\over c^2}+2\right]cQ_k\,,\cr
{\d Q_\kappa\over\d r}&={\kappa\over r}Q_\kappa-{1\over c}(E-V)P_k\,,\cr
}$$
which can be found in \cite{zabloudil} (eq. 8.12 and 8.13), where they have one
$c$
hidden in $Q_\kappa=crf_\kappa$ and use Rydberg atomic units, so they have $1$ instead of $2$ in the square bracket.
It can be found in \cite{bachelet} as well, they use Hartree atomic units, but
have a different notation $G_\kappa\equiv P_\kappa$ and $F_\kappa\equiv
Q_\kappa$, also they made a substitution $c={1\over\alpha}$.


Some authors also use $\epsilon\equiv E$.
