\chapter{Finite Groups}

\section{Theory}

Definition of a group:

\halign{#\quad &#\hfil\cr
I1 & $x,y \in G  \Rightarrow  xy \in G$ \cr
I2 & there exist $e$ such that $ex=xe=x$ for each $x\in G$ \cr
I3 & there exist $x^{-1}$ such that $xx^{-1}=x^{-1}x=e$ for each $x\in G$ \cr
I4 & $(xy)z=x(yz)$ for each $x,y,z\in G$ \cr
}

\medskip

Every finite group is isomorphic to a subgroup of $S_n$ (permutations).

Representation:

Set of linear operators T(x) (for each $x\in G$ there is one $T(x)$)
$$T(x)T(y)=T(xy),\quad T(e)=1\,.$$

T(x) fulfils all the group axioms I1, I2, I3, I4.

The representation T(x) is said to be 'faithful' if there is a one-to-one
relationship between T(x) and x (an isomorphism).

Equivalent representations $D_1$ and $D_2$: there exist $S$ such that
$D_2=SD_1(x)S^{-1}$ for each $x\in G$.

Reducible representation $D(x)$: there exist 
$$D'(x)=SD(x)S^{-1}=\matd{D_1'}{D_2'}$$ for each $x\in G$.
We say that $D'$ is a direct sum of $D_1'$ and $D_2'$: $D'=D_1'\oplus D_2'$.

Irreducible representation: is not reducible.

Conjugate element: $x$ is conjugate to $y$ ($x\sim y$) if there exist $c$ such that:
$$x=cyc^{-1}$$

if $x\sim y$ and $y\sim z$ then $x\sim z$.

conjugate class: elements which are all conjugate to each other

no element may belong to more than one class $\Rightarrow$ every group may be
broken up into separate classes.

character $\chi$ of the representation $D(x)$: set of numbers $\chi(x)$ as the
group element $x$ runs through the group $\chi(x)=\Tr D(x)$

Equivalent representations have the same character:
$$\chi'(x)=\Tr D'(x)=\Tr SD(x)S^{-1}=\Tr D(x)=\chi(x)$$

Representations having the same character are equivalent.

Proof: Characters can be thought of as elements of a q-dimensional vector space
where q is the number of conjugacy classes. Using the orthogonality relations
derived above, we find that the q characters for the q inequivalent irreducible
representations forms a basis set. Also, according to Maschke's theorem, both
representations can be expressed as the direct sum of irreducible
representations. Since the character of the direct sum of representations is
the sum of their characters, from linear algebra, we see they are equivalent.

All the elements in the same class have the same character.

Maschke's theorem: for finite groups, every class of equivalent representations
contains unitary representations. The theorem is also true for most infinite
groups on interest in physics.

Let $T$ be a reducible representation, then:
$$T=m_1T^{(1)} \oplus m_2T^{(2)} \oplus m_3T^{(3)}\oplus \cdots$$
where $T^{(1)}$, $T^{(2)}$, $T^{(3)}$ \dots are all the inequivalent irreducible
representations and $m_\alpha$ ($\alpha=1,2,3,\dots$) gives the number of times
that the irreducible representation $T^{(\alpha)}$ occurs in the reduction.

Similar relation holds for group characters:
$$\chi=m_1\chi^{(1)} + m_2\chi^{(2)} + m_3\chi^{(3)} + \cdots$$
and it can be shown 4.28, p. 63:
$$m_\alpha={1\over g}\sum_{x\in G} \chi^{(\alpha)*}(x)\chi(x)=
{1\over g}\sum_{p} c_p\chi^{(\alpha)*}_p\chi_p$$
where $c_p$ is the number of elements in the class $p$, $g$ is the number of
elements in $G$ (the order of the group).

Number of irreducible representations $=$ the number of classes.


\section{Point groups}

Karel+Ondra

\section{Construction and usage of the character table}

Ondra

Construction of the character table, 4.15, p 67

\section{Applications of finite groups}

Multiplets (Ondra), transitions (Karel) and vibrations (Karel)




\chapter{Continuous Groups}

\section{Lie groups+algebras}

Karel

\section{Young diagrams}

Karel
