\chapter{Finite Groups}

\section{Theory}

Definition of a group:

\halign{#\quad &#\hfil\cr
I1 & $x,y \in G  \Rightarrow  xy \in G$ \cr
I2 & there exist $e$ such that $ex=xe=x$ for each $x\in G$ \cr
I3 & there exist $x^{-1}$ such that $xx^{-1}=x^{-1}x=e$ for each $x\in G$ \cr
I4 & $(xy)z=x(yz)$ for each $x,y,z\in G$ \cr
}

\medskip

Every finite group is isomorphic to a subgroup of $S_n$ (permutations).

Representation:

Set of linear operators T(x) (for each $x\in G$ there is one $T(x)$)
$$T(x)T(y)=T(xy),\quad T(e)=1\,.$$

T(x) fulfils all the group axioms I1, I2, I3, I4.

The representation T(x) is said to be 'faithful' if there is a one-to-one
relationship between T(x) and x (an isomorphism).

Equivalent representations $D_1$ and $D_2$: there exist $S$ such that
$D_2=SD_1(x)S^{-1}$ for each $x\in G$.

Reducible representation $D(x)$: there exist 
$$D'(x)=SD(x)S^{-1}=\matd{D_1'}{D_2'}$$ for each $x\in G$.
We say that $D'$ is a direct sum of $D_1'$ and $D_2'$: $D'=D_1'\oplus D_2'$.

Irreducible representation: is not reducible.

Conjugate element: $x$ is conjugate to $y$ ($x\sim y$) if there exist $c$ such that:
$$x=cyc^{-1}$$

if $x\sim y$ and $y\sim z$ then $x\sim z$.

conjugate class: elements which are all conjugate to each other

no element may belong to more than one class $\Rightarrow$ every group may be
broken up into separate classes.

character $\chi$ of the representation $D(x)$: set of numbers $\chi(x)$ as the
group element $x$ runs through the group $\chi(x)=\Tr D(x)$

Equivalent representations have the same character:
$$\chi'(x)=\Tr D'(x)=\Tr SD(x)S^{-1}=\Tr D(x)=\chi(x)$$

Representations having the same character are equivalent.

Proof: Characters can be thought of as elements of a q-dimensional vector space
where q is the number of conjugacy classes. Using the orthogonality relations
derived above, we find that the q characters for the q inequivalent irreducible
representations forms a basis set. Also, according to Maschke's theorem, both
representations can be expressed as the direct sum of irreducible
representations. Since the character of the direct sum of representations is
the sum of their characters, from linear algebra, we see they are equivalent.

All the elements in the same class have the same character.

Maschke's theorem: for finite groups, every class of equivalent representations
contains unitary representations. The theorem is also true for most infinite
groups on interest in physics.

Let $T$ be a reducible representation, then:
$$T=m_1T^{(1)} \oplus m_2T^{(2)} \oplus m_3T^{(3)}\oplus \cdots$$
where $T^{(1)}$, $T^{(2)}$, $T^{(3)}$ \dots are all the inequivalent irreducible
representations and $m_\alpha$ ($\alpha=1,2,3,\dots$) gives the number of times
that the irreducible representation $T^{(\alpha)}$ occurs in the reduction.

Similar relation holds for group characters:
$$\chi=m_1\chi^{(1)} + m_2\chi^{(2)} + m_3\chi^{(3)} + \cdots$$
and it can be shown \cite{elliott} (eq. 4.28, page 63):
$$m_\alpha={1\over g}\sum_{x\in G} \chi^{(\alpha)*}(x)\chi(x)=
{1\over g}\sum_{p} c_p\chi^{(\alpha)*}_p\chi_p$$
where $c_p$ is the number of elements in the class $p$, $g$ is the number of
elements in $G$ (the order of the group).

Number of irreducible representations $=$ the number of classes.


\section{Point groups}

We are concerned with all subgroups of $O(3)$, which leave a crystal lattice
invariant. Those are just 7: $S_2$ (triclinic), $C_{2h}$ (monoclinic), $D_{2h}$
(orthorhombic), $D_{3d}$ (rhombohedral), $D_{4h}$ (tetragonal), $D_{6h}$
(hexagonal) and $O_{h}$ (cubic).

For simple monoatomic crystals with one atom per unit cell these seven are the
only possible crystallographic point groups. For more complicated crystals with
a molecule or an arrangement of atoms in the unit cell, the symmetry will be
reduced to the subgroup which leaves the unit cell invariant.

The complete list of all possible crystallographic point groups will therefore
be given by the above seven together with all their subgroups:

\halign{$#$\quad\quad &$#$\hfil\cr
S_2    & C_{1h}, S_2 \cr
C_{2h} & C_2, C_{1h}, C_{2h} \cr
D_{2h} & D_2, C_{2v}, D_{2h} \cr
D_{3d} & C_3, S_6, D_3, C_{3v}, D_{3d} \cr
D_{4h} & C_4, S_4, C_{4h}, D_4, C_{4v}, D_{2d}, D_{4h} \cr
D_{6h} & C_3, S_6, D_3, C_{3v}, D_{3d}, C_6, C_{3h}, C_{6h}, D_6, C_{6v},
    D_{3h}, D_{6h} \cr
O_h    & T, T_h, O, T_d, O_h \cr
}

There are 37 subgroups together, but $D_{3d}$ and $D_{6h}$ contain 5 equal
subgroups, so together we get 32 distinct subgroups. Groups, which might at
first sight appear to be missing from the list are $C_{1v}$, $D_1$, $D_{1h}$,
$S_1$, and $S_3$, but these are the same as $C_{1h}$, $C_2$, $C_{2v}$, $C_{1h}$
and $C_{3h}$ respectively.

These groups are isomorphic: 

$C_{1h}$, $S_2$, $C_2$

$S_4$, $C_4$

$S_6$, $C_{3h}$, $C_6$

$C_{2h}$, $C_{2v}$, $D_2$

$C_{3v}$, $D_3$

$D_{2d}$, $C_{4v}$, $D_4$

$D_{3d}$, $D_{3h}$, $C_{6v}$, $D_6$

$T_d, O$

\section{Construction and usage of the character table}

Ondra

Construction of the character table \cite{elliott} (sec. 4.15, page 67):

\section{Applications of finite groups}

Multiplets (Ondra), transitions (Karel) and vibrations (Karel)




\chapter{Continuous Groups}

\section{Lie groups+algebras}

(Karel)

A continuous group with metrics is a Lie group (did I forgot anything?),
usually a subgroup of $GL(n)$ is meant, a linear Lie group (i.e. matrices). In
a compact (?) linear Lie group $G$, any element can be generated by
$\exp(-tA)$ where $t\in R$ and $A$ (not necessarilly $\in G$ (?)) is one of
the generators. Typical example: for $G=SO(3)$ there are three generators,
$iA_x$, $iA_y$, $iA_z$, where $A_x$ is the rotation by $\pi/2$ around
$x$--axis in $R^3$. The generators form a vector space (here the linear span
of $iA_x$, $iA_y$, $iA_z$) with an additional operation of
commutation. This structure is closed and it is called the Lie algebra of the
group $G$. The commutation relations between the generators fully specify the
Lie algebra. E.g. $[iA_x,iA_y]=iA_z$ (or minus?) and the two other ones.

This is a great simplification because a continuous (infinite) group was thus
mapped on a vector space, the algebra, where it suffices to look at the basis
elements, the generator. The net effect is that we have to watch only three
objects instead of infinitely many in the example above.



\section{Young diagrams}

(Karel)

YD is a systematic method to find all IRs of any symmetric group $S_n$
(permutations of an $n$-element set). The idea:

$\bullet$ find all equivalence classes of $S_n$

$\bullet$ assign an IR to each of them

\noindent
Details are explained in Sternberg or in the lecture notes of J. Niederle. 





