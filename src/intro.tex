\def\em{\it}

\chapter{Introduction}

The outcome of this thesis is an algorithm solving the radial Dirac equation
(together with the corresponding computer code) in a specific case, as a part 
of a particular method for electronic structure calculations in solid-state
physics.

The electronic structure calculations are essential for any theoretical study
of materials, which is currently an extremely active field of research 
and is often denoted as {\em computational material science}.
In past few decades, electronic structure calculations made a significant 
contribution to our understanding of material properties,
using a large variety of continuously developing methods and approaches.

This work is related to one of up-to-date {\em ab-initio pseudopotential} 
methods based on the {\em density-functional theory} 
(DFT)\cite{wikidft,pickett},
particularly to the pseudopotential generating process within the 
{\em all-electron pseudopotential} method
\cite{vackarAEPP1,vackarAEPP2}.

{\em Ab-initio method} means that the method do not require any empirical 
parameters as an input to the calculation --- being based on equations 
derived directly from theoretical principles, with no need for 
experimental input data.

The standard result of the particular calculation within DFT is the total 
energy of the given system and the electronic charge density (or the wave 
function of electronic states that, integrated over the space and occupied 
states, form the charge density). Within DFT, the electronic charge density 
is a key quantity containing complete information about the system. 
This information allows us to answer almost any question we might ask about 
the solid: 
the structure of electronic states provides information about thermal and
electrical conductivity, e.g. metallic or semiconductive behavior, 
optical or X-ray emission/absorption spectra and scattering properties;
minimization of the total energy with respect to atomic positions provides 
an equilibrium geometry; derivatives of the total energy gives bulk modulus 
and elastic constants; spin structure of electronic states and derivatives 
of the total energy with respect to an external fields can provide dielectric
or magnetic properties of the solid, etc.

At present, several more-or-less ab-initio computational methods withing DFT 
are widely used in solid state physics. Every method usually has it's own domain 
in which it excels. The success of a method is determined by many factors, 
including its usage in particular computer codes and their efficiency, 
accuracy etc. And new methods can emerge in the future. It is not the
aim of this thesis to compare all the methods available. It will suffice 
to say that in present, the majority of calculations are based on the
local-density approximation (or its extensions) to DFT, which 
we review very briefly now\cite{pickett}.
It leads to the following Kohn-Sham equations (for $N$ electrons):
$$\left\{ -{1\over2m}\nabla^2+
\sum_{n=1}^N V_n^{\rm ion}({\bf x}-{\bf X}_n)+
V_H({\bf x})+V_{XC}({\bf x})
\right\}\psi_i({\bf x})=\epsilon_i\psi_i({\bf x})\,,\no{kohn-sham}$$
where the charge density is
$$\rho=\sum_{i=1}^N |\psi_i|^2\,,$$
the Hartree potential $V_H$ is given by
$$\nabla^2V_H=-4\pi\rho\,,$$
and the exchange-correlation potential $V_{XC}$ is just a given function of the
charge density $\rho$. The $V_n^{\rm ion}$ is the electrostatic 
potential of atomic nuclei. Within the family of up-to-date
pseudopotential methods, it is constructed from 
nonlocal pseudopotentials. There are several ways how to do that. 

These equations need to be solved self-consistently, so that the charge
density, which is used for the construction of the Hartree and
exchange-correlation potential is the same as the charge density calculated from
the equations, see also fig. \refn{schema01}.

%\psfigw{../images/schema01.eps}{schema01}{Self-consistency cycle}{6cm}

Numerous methods have been developed to solve the
resulting system of one-particle Kohn-Sham equations \rno{kohn-sham}. 
Depending on the basis set used and other features of the particular method, 
some of them (in fact: almost all of them) need the radial part of the equation
to be solved. 
For heavy atoms (approximately, starting at atomic numbers around 40-50) 
the equations \rno{kohn-sham} are considerably inaccurate due to
the non-negligible relativistic effects, especially for the core electronic states
bounded with relatively high energies. 
For this reason, it is desirable to use a relativistic Dirac
equation instead, which is the aim of this thesis.

In the present thesis we first derive and show how to solve the radial Schr\"odinger
equation for a given energy. Then we move to the relativistic case, Dirac
equation, and do the same. Next we show how to solve the eigenproblem, that is,
how to determine the energies, for which the solution is normalizable. Then we
say something about the computer implementation together with some results of
the program. In the appendix we define atomic units (which is sometimes a
little confusing issue) and we derive spin-angular functions including some of their
properties we need in our work.
